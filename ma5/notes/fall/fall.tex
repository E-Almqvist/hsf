\documentclass{article}

\usepackage[margin=2cm]{geometry}
\usepackage{titlesec}
\usepackage{titling}
\usepackage[hidelinks]{hyperref}
\usepackage{multicol}
\usepackage{amsmath}
\usepackage{amsfonts}
\usepackage{amssymb}
\usepackage{braket}

\titleformat{\section}
{\Large\bfseries}
{}
{0em}
{}[\titlerule]

\titleformat{\subsection}
{\large\bfseries}
{}
{0em}
{}
\titlespacing{\subsection}
{0em}{2em}{.4em}

\titleformat{\subsubsection}[runin]
{\bfseries}
{}
{0em}
{}
\titlespacing{\subsubsection}
{0em}{2em}{1em}

\renewcommand{\maketitle}{
	\begin{center}
		{\huge\bfseries\thetitle}\\
		\vspace{1em}
		{\Large\theauthor} \\
		\vspace{1em}
		elalmqvist@gmail.com --- \url{https://wych.dev}
	\end{center}
}


\begin{document}

\title{Anteckningar 2022-05-05}
\author{Elias Almqvist}

\maketitle
\newpage

\section{Fallande kropp i atmosfär}
Något som faller har tyngdaccelerationen $g$ i $\frac{m}{s^2}$ och en massa $m$ i kg. $v$ är dess hastighet i $\frac{m}{s}$ och $F$ är kraften som verkas på den i newton. Den har dessutom en luftmotståndsfunktion: $\gamma(v)$. Låt positiva ($v > 0$) gå uppåt:
$$
F_g = -mg
$$
$$
F_\gamma = \gamma(v)
$$
$$
\implies F = \Delta F = F_g - F_\gamma = -mg - \gamma(v)
$$

För att få den resulterande accelerationen $a$ dividerar vi med $m$:
$$
a = \frac{F}{m} \because F = ma
$$
$$
a = \frac{-mg - \gamma(v)}{m} = -\left(g + \frac{\gamma(v)}{m}\right)
$$

Vi kan också skriva om $a$:
$$
a = \frac{dv}{dt} = \dot{v} = -\left(g + \frac{\gamma(v)}{m}\right), \quad +\frac{\gamma(v)}{m}
$$
$$
\therefore \dot{v} + \frac{1}{m} \gamma(v) = -g
$$

Nu varierar ekvationen beroende på vad $\gamma(v)$ är för något. Men vi kan anta att $F_\gamma \propto v$ vilket ger:
$$
\dot{v} + \frac{1}{m} kv = -g \quad | \quad k \in \mathbb{R}
$$
$$
\implies v_h = Ce^{-\frac{k}{m} t} \quad | \quad C \in \mathbb{R}
$$
$$
\implies v_p = \frac{-mg}{k}
$$
$$
\therefore v_a = v_p + v_h = Ce^{-\frac{k}{m} t} - \frac{mg}{k}
$$

Vi har nu den en generell lösning. Behöver bara lösa vad $C$ är givet ett villkor. etc etc. \\
Exempel:
$$
v(0) = 0 \implies C - \frac{mg}{k} = 0 \implies C = \frac{mg}{k}
$$
$$
\therefore v = \frac{mg}{k} \left( e^{\frac{-k}{m}t} - 1 \right)
$$

\end{document}
