\documentclass{article}

\usepackage[margin=2cm]{geometry}
\usepackage{titlesec}
\usepackage{titling}
\usepackage[hidelinks]{hyperref}
\usepackage{multicol}
\usepackage{amsmath}
\usepackage{amsfonts}

\titleformat{\section}
{\Large\bfseries}
{}
{0em}
{}[\titlerule]

\titleformat{\subsection}
{\large\bfseries}
{}
{0em}
{}
\titlespacing{\subsection}
{0em}{0em}{.4em}

\titleformat{\subsubsection}[runin]
{\bfseries}
{}
{0em}
{}[:]

\titlespacing{\subsubsection}
{0em}{0em}{1em}

\renewcommand{\maketitle}{
	\begin{center}
		{\huge\bfseries\thetitle}\\
		\vspace{1em}
		{\Large\theauthor} \\
		\vspace{1em}
		elalmqvist@gmail.com --- \url{https://wych.dev}
	\end{center}
}

% Wave function
\newcommand{\wavefun}{\psi_n(x)}

% Schrödingers equation
\newcommand{\shrodequ}{E_n \psi_n(x) = - \frac{\hbar^2}{2m} \frac{d^2 \psi_n}{dx^2}}

% Probability density for the particle
\newcommand{\shrodprob}{|\psi_n(x)|^2}

% Time factor thing
\newcommand{\shrodtime}{e^{-i \frac{E_n}{\hbar} t}}

% Full Schrödinger equation
\newcommand{\fullshrodequ}{\Psi_n(x, t) = \psi_n(x) \shrodtime}

% Full Schrödinger equation prob density
\newcommand{\fullshrodprob}{|\Psi_n(x,t)|^2}


\begin{document}

\title{Schrödinger ekvationen (partikel i låda)}
\author{Elias Almqvist}

\maketitle
\newpage

\section{Uppgiftbeskrivning (taget från dokumentet)}
En partikel i en låda är en utav de första tillämpningarna man stöter på när man lär sig om kvantfysik. Man betraktar då en partikel (t.ex. en elektron) som befinner sig i en låda med oändligt höga väggar.
För detta undersöker man partikelns vågfunktion $\wavefun$. Vågfunktionen är i allmänhet en komplex funktion,
dvs den har både en realdel och en imaginärdel. Vågfunktionens absolutbelopp i kvadrat, $\shrodprob$, representerar täthetsfunktionen för att partikeln skall befinna sig vid läge $x$ i lådan. Om partikeln befinner sig i ett så
kallat energiegentillstånd så uppfyller den den tidsoberoende Schrödinger ekvationen:
\begin{equation} \label{shrodequ}
	\shrodequ
\end{equation}

där $E_n$ är partikelns energi, $\hbar = \frac{h}{2\pi}$ och $m$ är partikelns massa.
Att lådans väggar är oändligt höga innebär att vågfunktionen också behöver uppfylla randvillkoren:
\begin{equation} \label{shrodequ_con1}
	\psi_n(0) = \psi_n(L) = 0 \quad \& \quad \psi_n'(0) = \psi_n'(L) = 0 
\end{equation}
Slutligen, eftersom $\shrodprob$ motsvarar sannolikhetstätheten för att partikeln skall befinna sig vid position $x$, så måste det gälla att:
\begin{equation} \label{shrodequ_con2}
	\int_0^L \shrodprob dx\ = 1.0
\end{equation}

\subsection{Uppgifter}
\begin{enumerate}
	\item Hitta de olika möjliga värden på $E_n$, och hitta motsvarande vågfunktioner $\wavefun$. 
	\item Visa grafer över motsvarande sannolikhetsfördelningar för att partikeln skall befinna sig vid olika positioner $x$.

	\item Partikelns fullständiga vågfunktion är egentligen även en funktion utav tiden. För en partikel som befinner sig i ett så kallat energiegentillstånd är den fullständiga vågfunktionen $\fullshrodequ$
		Dock innebär den extra faktorn $\shrodtime$ inte någon intressant tidsutveckling av sannolikhetsfördelningen eftersom $|\Psi(x, t)|^2 = |\psi_n(x) \shrodtime|^2 = \shrodprob$. Intressantare blir det om en partikel befinner sig i en superposition av energiegentillstånd, tex: $$\Psi(x, t) = A(\psi_1(x)e^{-i \frac{E_1}{\hbar}t} + \psi_2(x) e^{-i \frac{E_2}{\hbar}t})$$ För denna vågfunktion, \emph{bestäm konstanten $A$ sådan att}: $$\int_0^L |\Psi(x, t)|^2 dx = 1.0$$ \emph{Undersök sedan hur sannolikheten att befinna sig i den vänstra delen $0 < x < \frac{L}{2}$, respektive högra $\frac{L}{2} < x < L$ delen av lådan}. Hitta alltså ett uttryck för: $$P(V, t) = \int_0^{\frac{L}{2}} \fullshrodprob dx$$ $$P(H, t) = \int_{\frac{L}{2}}^L \fullshrodprob dx$$

	\item Gör sedan samma sak för superpositionen av energiegentillstånden 1 och 3: $$\Psi(x, t) = A\left(\psi_1(x)e^{-i \frac{E_1}{\hbar}t} + \psi_3(x)e^{-i \frac{E_3}{\hbar}t}\right)$$ \emph{På vilket sätt skiljer de sig? Kan du förklara varför?}
\end{enumerate}

\newpage

\section{Uppgiftlösningar}
\subsection{1}

Enligt Schrödingers ekvation får vi: $\shrodequ$ där $\hbar = \frac{h}{2\pi}$ vilket vi kan substituera i ekvationen och vi får följande:

$$
	\shrodequ, \quad \left[\hbar / \frac{h}{2\pi}\right]
$$
$$
	E_n\psi_n(x) = - \frac{\left(\frac{h}{2\pi}\right)^2}{2m} \frac{d^2 \psi_n}{dx^2} = - \frac{h^2}{8 \pi^2 m} \left(\frac{d^2 \psi_n}{dx^2}\right)
$$

där $h$ är Plancks konstant och $m$ är partikelns massa. Väljer därmed att förenkla uttrycket genom att byta ut konstanterna till en variabel (givet att $k = \frac{h^2}{8 \pi ^2 m}$):
$$
	E_n\psi_n(x) = - \frac{h^2}{8 \pi ^2 m} \left(\frac{d^2 \psi_n}{dx^2}\right), \quad \left[\frac{h^2}{8 \pi ^2 m}/k\right]
$$
$$
	E_n\psi_n(x) = -k\left(\frac{d^2 \psi_n}{dx^2}\right), \quad + HL
$$
$$
	E_n\psi_n(x) + k\left(\frac{d^2 \psi_n}{dx^2}\right) = 0
$$

Väljer att skriva om differentialekvationen utan Leibnizs notation och vi får:
$$
	E_n\psi_n + k \psi_n'' = 0, \quad /E_n 
$$
$$
	\psi_n'' + \frac{E_n}{k}\psi_n = 0 
$$

Vet att differentialekvationer av andra ordningen har lösningen $y=e^{\lambda x}$ och vi kan därmed beräkna $\lambda$ för vår differentialekvation genom den karakteristiska ekvationen:
$$
	\lambda^2 + a\lambda + b = 0
$$

där $a$ och $b$ är koefficienterna framför respektive "funktion". I vårt fall är $a=0$ och $b = \frac{E_n}{k}$ och vi får därmed den karakteristiska ekvationen:
$$
	\lambda^2 + \frac{E_n}{k} = 0, \quad PQ	
$$
$$
	\lambda = \pm \sqrt{\frac{E_n}{k}}i
$$

Då rötterna för den karakteristiska ekvationen är komplexa ($\in \mathbb{C}$) får vi den \emph{allmäna lösningen}:
$$
	\psi_n(x) = e^{ax}\left(C \cos bx + D \sin bx\right) \quad | \quad C,D \in \mathbb{R}, \quad \lambda = a + bi
$$
$$
	\psi_n(x) = e^{0}\left( C \cos \pm\sqrt{\frac{E_n}{k}}x + D \sin \pm\sqrt{\frac{E_n}{k}}x \right) 
$$
\begin{equation} \label{psi_gen}
	\psi_n(x) = C \cos \left(\pm\sqrt{\frac{E_n}{k}}x\right) + D \sin \left(\pm\sqrt{\frac{E_n}{k}}x\right)
\end{equation}

Schrödinger ekvationen lyder också att vågfunktionen skall följa både ekvation \ref{shrodequ_con1} och \ref{shrodequ_con2} vilket ger:
$$
	\begin{cases}
		\int_0^L \shrodprob dx\ = 1.0, & P(1) \\
		\psi_n(0) = \psi_n(L) = 0, & P(2) \\
		\psi_n'(0) = \psi_n'(L) = 0, & P(3) 
	\end{cases}
$$

Givet att $P(1)$ implicerar det att vågfunktion $\psi_n(x)$ area mellan $0$ och $L$ är $1$ och $P(2)$ samt $P(3)$ gäller vilket ger att det är en stående våg och den har därmed ett visst antal våglängder ($\lambda$) i relation till antal noder ($n$). 


\end{document}
