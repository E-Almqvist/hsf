\documentclass{article}

\usepackage[margin=2cm]{geometry}
\usepackage{titlesec}
\usepackage{titling}
\usepackage[hidelinks]{hyperref}
\usepackage{multicol}

\titleformat{\section}
{\Large\bfseries}
{}
{0em}
{}[\titlerule]

\titleformat{\subsection}
{\large\bfseries}
{}
{0em}
{}
\titlespacing{\subsection}
{0em}{0em}{.4em}

\titleformat{\subsubsection}[runin]
{\bfseries}
{}
{0em}
{}[:]

\titlespacing{\subsubsection}
{0em}{0em}{1em}

\renewcommand{\maketitle}{
	\begin{center}
		{\huge\bfseries\thetitle}\\
		\vspace{1em}
		{\Large\theauthor} \\
		\vspace{1em}
		elalmqvist@gmail.com --- \url{https://wych.dev}
	\end{center}
}

% Wave function
\newcommand{\wavefun}{\psi_n(x)}

% Schrödingers equation
\newcommand{\shrodequ}{E_n \psi_n(x) = - \frac{\hbar}{2m} \frac{d^2 \psi_n}{dx^2}}

% Probability density for the particle
\newcommand{\shrodprob}{|\psi_n(x)|^2}

% Time factor thing
\newcommand{\shrodtime}{e^{-i \frac{E_n}{\hbar} t}}

% Full Schrödinger equation
\newcommand{\fullshrodequ}{\Psi_n(x, t) = \psi_n(x) \shrodtime}

% Full Schrödinger equation prob density
\newcommand{\fullshrodprob}{|\Psi_n(x,t)|^2}


\begin{document}

\title{Schrödinger ekvationen (partikel i låda)}
\author{Elias Almqvist}

\maketitle
\newpage

\section{Uppgiftbeskrivning (taget från dokumentet)}
En partikel i en låda är en utav de första tillämpningarna man stöter på när man lär sig om kvantfysik. Man betraktar då en partikel (t.ex. en elektron) som befinner sig i en låda med oändligt höga väggar.
För detta undersöker man partikelns vågfunktion $\wavefun$. Vågfunktionen är i allmänhet en komplex funktion,
dvs den har både en realdel och en imaginärdel. Vågfunktionens absolutbelopp i kvadrat, $\shrodprob$, represen-
terar täthetsfunktionen för att partikeln skall befinna sig vid läge $x$ i lådan. Om partikeln befinner sig i ett så
kallat energiegentillstånd så uppfyller den den tidsoberoende Schrödinger ekvationen:
\begin{equation}
	\shrodequ
\end{equation}

där $E_n$ är partikelns energi, $\hbar = \frac{h}{2\pi}$ och $m$ är partikelns massa.
Att lådans väggar är oändligt höga innebär att vågfunktionen också behöver uppfylla randvillkoren:
$$
\psi_n(0) = \psi_n(L) = 0 \quad \& \quad \psi_n'(0) = \psi_n'(L) = 0 
$$
Slutligen, eftersom $\shrodprob$ motsvarar sannolikhetstätheten för att partikeln skall befinna sig vid position $x$, så måste det gälla att:
$$
\int_0^L \shrodprob dx\ = 1.0
$$

\subsection{Uppgifter}
\begin{enumerate}
	\item Hitta de olika möjliga värden på $E_n$, och hitta motsvarande vågfunktioner $\wavefun$. Visa grafer över motsvarande sannolikhetsfördelningar för att partikeln skall befinna sig vid olika positioner $x$.

	\item Partikelns fullständiga vågfunktion är egentligen även en funktion utav tiden. För en partikel som befinner sig i ett så kallat energiegentillstånd är den fullständiga vågfunktionen $\fullshrodequ$
		Dock innebär den extra faktorn $\shrodtime$ inte någon intressant tidsutveckling av sannolikhetsfördelningen eftersom $|\Psi(x, t)|^2 = |\psi_n(x) \shrodtime|^2 = \shrodprob$. Intressantare blir det om en partikel befinner sig i en superposition av energiegentillstånd, tex: $$\Psi(x, t) = A(\psi_1(x)e^{-i \frac{E_1}{\hbar}t} + \psi_2(x) e^{-i \frac{E_2}{\hbar}t})$$ För denna vågfunktion, \emph{bestäm konstanten $A$ sådan att}: $$\int_0^L |\Psi(x, t)|^2 dx = 1.0$$ \emph{Undersök sedan hur sannolikheten att befinna sig i den vänstra delen $0 < x < \frac{L}{2}$, respektive högra $\frac{L}{2} < x < L$ delen av lådan}. Hitta alltså ett uttryck för: $$P(V, t) = \int_0^{\frac{L}{2}} \fullshrodprob dx$$ $$P(H, t) = \int_{\frac{L}{2}}^L \fullshrodprob dx$$

	\item Gör sedan samma sak för superpositionen av energiegentillstånden 1 och 3: $$\Psi(x, t) = A(\psi_1(x)e^{-i \frac{E_1}{\hbar}t} + \psi_3(x)e^{-i \frac{E_3}{\hbar}t})$$ \emph{På vilket sätt skiljer de sig? Kan du förklara varför?}
\end{enumerate}

\newpage

\section{Uppgiftlösningar}
\subsection{1}


\end{document}
