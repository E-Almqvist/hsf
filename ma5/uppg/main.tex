\documentclass{article}

\usepackage[margin=2cm]{geometry}
\usepackage{titlesec}
\usepackage{titling}
\usepackage[hidelinks]{hyperref}
\usepackage{multicol}
\usepackage{amsmath}
\usepackage{amsfonts}
\usepackage{amssymb}
\usepackage{braket}

\titleformat{\section}
{\Large\bfseries}
{}
{0em}
{}[\titlerule]

\titleformat{\subsection}
{\large\bfseries}
{}
{0em}
{}
\titlespacing{\subsection}
{0em}{2em}{.4em}

\titleformat{\subsubsection}[runin]
{\bfseries}
{}
{0em}
{}
\titlespacing{\subsubsection}
{0em}{2em}{1em}

\renewcommand{\maketitle}{
	\begin{center}
		{\huge\bfseries\thetitle}\\
		\vspace{1em}
		{\Large\theauthor} \\
		\vspace{1em}
		elalmqvist@gmail.com --- \url{https://wych.dev}
	\end{center}
}

% Wave function
\newcommand{\wavefun}{\psi_n(x)}

% Schrödingers equation
\newcommand{\shrodequ}{E_n \psi_n(x) = - \frac{\hbar^2}{2m} \frac{d^2 \psi_n}{dx^2}}

% Probability density for the particle
\newcommand{\shrodprob}{|\psi_n(x)|^2}

% Time factor thing
\newcommand{\shrodtime}{e^{-i \frac{E_n}{\hbar} t}}

% Full Schrödinger equation
\newcommand{\fullshrodequ}{\Psi_n(x, t) = \psi_n(x) \shrodtime}

% Full Schrödinger equation prob density
\newcommand{\fullshrodprob}{|\Psi_n(x,t)|^2}


\begin{document}

\title{Schrödingers ekvation (partikel i låda)}
\author{Elias Almqvist}

\maketitle
\newpage

\section{Notation \& syntax}
\subsection{Operationer}
Ekvationer följda med ett "$,$" och sedan ett uttryck menas att uttrycket utsätts på både höger och vänster led. Uttrycket kan också vara en formel som till exempel "$PQ$" eller "$\text{trig-ettan}$". \emph{Operationer utförs ej på villkor}. Exempel:
$$
2x = 8, \quad /2
$$
$$
x = 4
$$

\subsection{Substitution}
$$
, \quad \left[ a / b \right]
$$

Menas att $a$ byts ut mot $b$ i ekvationen vänster om kommatecknet.

\subsection{Gruppering}
$$
\left\{\text{\emph{uttryck}}\right\}
$$

Menas att allt inom måsvingarna är grupperad och skild från andra uttryck. Kan enbart utföras funktionella operationer på gruppen såsom $\frac{d}{dx}$ eller $\int$.

\newpage
\section{Uppgiftbeskrivning (taget från dokumentet)}
En partikel i en låda är en utav de första tillämpningarna man stöter på när man lär sig om kvantfysik. Man betraktar då en partikel (t.ex. en elektron) som befinner sig i en låda med oändligt höga väggar.
För detta undersöker man partikelns vågfunktion $\wavefun$. Vågfunktionen är i allmänhet en komplex funktion,
dvs den har både en realdel och en imaginärdel. Vågfunktionens absolutbelopp i kvadrat, $\shrodprob$, representerar täthetsfunktionen för att partikeln skall befinna sig vid läge $x$ i lådan. Om partikeln befinner sig i ett så
kallat energiegentillstånd så uppfyller den den tidsoberoende Schrödinger ekvationen:
\begin{equation} \label{shrodequ}
	\shrodequ
\end{equation}

där $E_n$ är partikelns energi, $\hbar = \frac{h}{2\pi}$ och $m$ är partikelns massa.
Att lådans väggar är oändligt höga innebär att vågfunktionen också behöver uppfylla randvillkoren:
\begin{equation} \label{shrodequ_con1}
	\psi_n(0) = \psi_n(L) = 0 \quad \& \quad \psi_n'(0) = \psi_n'(L) = 0 
\end{equation}
Slutligen, eftersom $\shrodprob$ motsvarar sannolikhetstätheten för att partikeln skall befinna sig vid position $x$, så måste det gälla att:
\begin{equation} \label{shrodequ_con2}
	\int_0^L \shrodprob dx\ = 1.0
\end{equation}

\subsection{Uppgifter}
\begin{enumerate}
	\item Hitta de olika möjliga värden på $E_n$, och hitta motsvarande vågfunktioner $\wavefun$. 
	\item Visa grafer över motsvarande sannolikhetsfördelningar för att partikeln skall befinna sig vid olika positioner $x$.

	\item Partikelns fullständiga vågfunktion är egentligen även en funktion utav tiden. För en partikel som befinner sig i ett så kallat energiegentillstånd är den fullständiga vågfunktionen $\fullshrodequ$
		Dock innebär den extra faktorn $\shrodtime$ inte någon intressant tidsutveckling av sannolikhetsfördelningen eftersom $|\Psi(x, t)|^2 = |\psi_n(x) \shrodtime|^2 = \shrodprob$. Intressantare blir det om en partikel befinner sig i en superposition av energiegentillstånd, tex: $$\Psi_{1, 2}(x, t) = A(\psi_1(x)e^{-i \frac{E_1}{\hbar}t} + \psi_2(x) e^{-i \frac{E_2}{\hbar}t})$$ För denna vågfunktion, \emph{bestäm konstanten $A$ sådan att}: $$\int_0^L |\Psi(x, t)|^2 dx = 1.0$$ \emph{Undersök sedan hur sannolikheten att befinna sig i den vänstra delen $0 < x < \frac{L}{2}$, respektive högra $\frac{L}{2} < x < L$ delen av lådan}. Hitta alltså ett uttryck för: $$P(V, t) = \int_0^{\frac{L}{2}} \fullshrodprob dx$$ $$P(H, t) = \int_{\frac{L}{2}}^L \fullshrodprob dx$$

	\item Gör sedan samma sak för superpositionen av energiegentillstånden 1 och 3: $$\Psi(x, t) = A\left(\psi_1(x)e^{-i \frac{E_1}{\hbar}t} + \psi_3(x)e^{-i \frac{E_3}{\hbar}t}\right)$$ \emph{På vilket sätt skiljer de sig? Kan du förklara varför?}
\end{enumerate}

\newpage

\section{Uppgiftlösningar}
\subsection{1}

Enligt Schrödingers ekvation får vi: $\shrodequ$ där $\hbar = \frac{h}{2\pi}$ vilket vi kan substituera i ekvationen och vi får följande:

$$
\shrodequ, \quad \left[\hbar / \frac{h}{2\pi}\right]
$$
$$
E_n\psi_n(x) = - \frac{\left(\frac{h}{2\pi}\right)^2}{2m} \frac{d^2 \psi_n}{dx^2} = - \frac{h^2}{8 \pi^2 m} \left(\frac{d^2 \psi_n}{dx^2}\right)
$$

där $h$ är Plancks konstant och $m$ är partikelns massa. Väljer därmed att förenkla uttrycket genom att byta ut konstanterna till en variabel (givet att $k = \frac{h^2}{8 \pi ^2 m}$):
$$
E_n\psi_n(x) = - \frac{h^2}{8 \pi ^2 m} \left(\frac{d^2 \psi_n}{dx^2}\right), \quad \left[\frac{h^2}{8 \pi ^2 m}/k\right]
$$
$$
E_n\psi_n(x) = -k\left(\frac{d^2 \psi_n}{dx^2}\right), \quad + HL
$$
$$
E_n\psi_n(x) + k\left(\frac{d^2 \psi_n}{dx^2}\right) = 0
$$

Väljer att skriva om differentialekvationen utan Leibnizs notation och vi får:
$$
E_n\psi_n + k \psi_n'' = 0, \quad /E_n 
$$
$$
\psi_n'' + \frac{E_n}{k}\psi_n = 0 
$$

Vet att differentialekvationer av andra ordningen har lösningen $y=e^{\lambda x}$ och vi kan därmed beräkna $\lambda$ för vår differentialekvation genom den karakteristiska ekvationen:
$$
\lambda^2 + a\lambda + b = 0
$$

där $a$ och $b$ är koefficienterna framför respektive "funktion". I vårt fall är $a=0$ och $b = \frac{E_n}{k}$ och vi får därmed den \textbf{\emph{karakteristiska ekvationen}}:
$$
\lambda^2 + \frac{E_n}{k} = 0, \quad PQ	
$$
$$
\lambda = \pm \sqrt{\frac{E_n}{k}}i
$$

Då rötterna för den karakteristiska ekvationen är komplexa ($\in \mathbb{C}$) får vi den \textbf{\emph{allmänna funktionen}}:
$$
\psi_n(x) = e^{ax}\left(C \cos bx + D \sin bx\right) \quad | \quad C,D \in \mathbb{R}, \quad \lambda = a + bi
$$
$$
\psi_n(x) = e^{0}\left( C \cos \pm\sqrt{\frac{E_n}{k}}x + D \sin \pm\sqrt{\frac{E_n}{k}}x \right) 
$$
\begin{equation} \label{psi_gen}
\therefore \psi_n(x) = C \cos \left(\sqrt{\frac{E_n}{k}}x\right) + D \sin \left(\sqrt{\frac{E_n}{k}}x\right)
\end{equation}

För att finna den \textbf{\emph{partikulära vågfunktionen}} måste vi ta hänsyn till villkoren \ref{shrodequ_con1} och \ref{shrodequ_con2} vilket ger:
$$
\begin{cases}
	\int_0^L \shrodprob dx\ = 1.0, & P(1) \\
	\psi_n(0) = \psi_n(L) = 0, & P(2) \\
	\psi_n'(0) = \psi_n'(L) = 0, & P(3) 
\end{cases}
$$

$P(2)$ och $P(3)$ lyder att sannolikheten att finna partikeln vid $x=0$ eller $x=L$ är $0.0$ vilket ger oss följande ekvation:
$$
\psi_n(0) = C \cos \left(\sqrt{\frac{E_n}{k}}0\right) + D \sin \left(\sqrt{\frac{E_n}{k}}0\right) = 0
$$
$$
\psi_n(0) = C \cos \left(0\right) + D \sin \left(0\right) = 0
$$
$$
\implies \psi_n(0) = D \sin \left(0\right) = 0 \implies C = 0
$$

Vi får därmed att $C=0$ om $P(2)$ skall gälla! Väljer att byta ut $k$ igen till dess ursprungliga uttryck och vi får:
$$
\psi_n(x) = D \sin \left(\sqrt{\frac{E_n}{k}} x \right), \quad \left[k/\frac{h^2}{8 \pi ^2 m}\right]
$$
$$
\therefore \psi_n(x) = D \sin \left( \sqrt{\frac{8 \pi^2 m E_n}{h^2}} x \right)
$$

$P(2)$ lyder också att vågfunktionen skall vara $0$ när $x=L$ och vi får därmed uttrycket:
$$
\psi_n(L) = D \sin \left( \sqrt{\frac{8 \pi^2 m E_n}{h^2}} L \right) = 0
$$
$$
\implies \sqrt{\frac{8 \pi^2 m E_n}{h^2}} L = 0 + \eta\pi \quad | \quad \eta \in \mathbb{N}, \quad /L
$$
\begin{equation} \label{coeff_eq}
\sqrt{\frac{8 \pi^2 m E_n}{h^2}} = \frac{\eta\pi}{L}
\end{equation}

Eftersom sannolikheten för att partikeln skall vara i lådan är alltid $1.0$ ger oss följande villkor $P(1)$ [\ref{shrodequ_con2}] och vi behöver därmed \emph{normalisera} vågfunktionen. Vi behöver alltså göra så att sannolikheten för att partikeln skall vara mellan $x=0$ och $x=L$ är $1$. Den är alltså alltid \emph{i lådan}. D.v.s. följande:
$$
\int_0^L \shrodprob dx\ = 1.0
$$
$$
\implies \shrodprob = \left( D\sin \sqrt{\frac{8 \pi^2 m E_n}{h^2}}x \right)^2
= D^2 \sin^2 \left( \sqrt{\frac{8 \pi^2 m E_n}{h^2}}x \right), \quad \left[ \sqrt{\frac{8 \pi^2 m E_n}{h^2}} / \frac{\eta\pi}{L} \right]
$$
$$
\shrodprob = D^2 \sin^2 \left( \frac{\eta\pi}{L} x \right)
$$
$$
\implies \int_0^L \shrodprob dx\ = D^2 \int_0^L \sin^2\left( \frac{\eta\pi}{L} x \right) dx\ = 1.0
$$

Vi behöver nu beräkna integralen och få fram dess uttryck. Vi använder oss därmed av \emph{u-substitution} och \emph{trigonometriska ettan}. 
$$
\textbf{\emph{Låt $u = \frac{\eta\pi}{L}x$}}
$$
$$
\int_0^L \sin^2\left( \frac{\eta\pi}{L} x \right) dx\ = \int_0^L \sin^2(u) dx\
$$
$$
\implies \frac{du}{dx} = \frac{\eta\pi}{L} \implies dx\ = \frac{L}{\eta\pi} du\
$$
$$
\implies \int_0^L \sin^2(u) dx\ = \frac{L}{\eta\pi} \int_0^L sin^2(u) du\
$$

Väljer att \emph{\textbf{skriva om och förenkla $sin^2(u)$}} och enlight den \emph{trigonometriska ettan} får vi:
$$
\cos(2u) = 1 - 2\sin^2 u, \quad \text{(dubbla vinkeln för cosinus)}, \quad +2\sin^2 u
$$
$$
\cos 2u + 2\sin^2 u = 1, \quad -\cos 2u
$$
$$
2\sin^2 u = 1 - \cos 2u, \quad /2
$$
$$
\implies \sin^2 u = \frac{1}{2} - \frac{1}{2}\cos 2u
$$

Vi kan nu stoppa in vår förenklade version av $sin^2(u)$ med något vi faktiskt kan integrera:
$$
\implies \int_0^L \shrodprob dx\ 
= D^2 \int_0^L \sin^2\left( \frac{\eta\pi}{L} x \right) dx\ 
= D^2 \frac{L}{\eta\pi} \int_0^L \left\{ \frac{1}{2} - \frac{1}{2}\cos(2u) \right\} du\ 
$$
$$
= D^2 \frac{L}{2\eta\pi} \int_0^L \left\{ 1 - \cos(2u) \right\} du\ 
= D^2 \frac{L}{2\eta\pi} \left( \int_0^L 1 du\ - \int_0^L \cos(2u) du\ \right)
$$
$$
= D^2 \frac{L}{2\eta\pi} \left[u - \int \cos(2u) du\ \right]_0^L
$$

$$
\emph{\textbf{Låt $\mu = 2u$}}
$$
$$
\implies \int \cos(2u) du\ = \int \cos(\mu) du\
$$
$$
\implies \frac{d\mu}{du} = 2 \implies du\ = \frac{1}{2} d\mu\
$$
$$
\int \cos(2u) du\ = \frac{1}{2} \int \cos(\mu) d\mu\ = -\frac{1}{2} \sin(\mu) + C
$$

$$
\implies \int_0^L \shrodprob dx\ 
= D^2 \frac{L}{2\eta\pi} \left[u - \int \cos(2u) du\ \right]_0^L
= D^2 \frac{L}{2\eta\pi} \left[ u + \frac{1}{2}\sin \mu \right]_0^L
= D^2 \frac{L}{2\eta\pi} \left[ u + \frac{1}{2}\sin 2u \right]_0^L
$$
$$
= D^2 \frac{L}{2\eta\pi} \left[ \frac{\eta\pi}{L}x + \frac{1}{2}\sin\left(2\frac{\eta\pi}{L}x\right) \right]_0^L
= D^2 \frac{L}{2\eta\pi} \left(\frac{\eta\pi}{L}L + \frac{1}{2}\sin\left(2\frac{\eta\pi}{L}L\right) \right)
= D^2 \frac{L}{2\eta\pi} \left(\eta\pi + \frac{1}{2}\sin\left(2\eta\pi\right) \right)
$$
$$
= D^2 \frac{L}{2\eta\pi} \left(\eta\pi + \frac{1}{2} \cdot 0 \right)
= D^2 \frac{L}{2\eta\pi} \cdot \eta\pi
= D^2 \frac{L}{2}
$$

Vi får därmed att $\int_0^L \shrodprob dx\ = D^2 \frac{L}{2} = 1.0$ vilket ger oss ekvationen:
$$
D^2 \frac{L}{2} = 1, \quad / \frac{L}{2}
$$
$$
D^2 = \frac{2}{L}, \quad \sqrt{\ }
$$
$$
\therefore D = \sqrt{ \frac{2}{L} }
$$
Därmed får vi slutligen vågfunktionen $\psi_n(x)$:
\begin{equation} \label{wavefun_part}
	\therefore \psi_n(x) = \sqrt{ \frac{2}{L} } \sin \left( \sqrt{\frac{8 \pi^2 m E_n}{h^2}} x \right)
\end{equation}
Det enda som återstår att att finna alla tillåtna energitillstånden ($E_n$). Sedan tidigare har vi följande uttryck [\ref{coeff_eq}]:
$$
\sqrt{\frac{8 \pi^2 m E_n}{h^2}} = \frac{\eta\pi}{L} \quad | \quad \eta \in \mathbb{N}
$$

Skriver om indexet $\eta$ som $n$ då de är samma variabel i detta fall. 
$$
\textbf{\emph{Låt $\eta = n$}}
$$
$$
\sqrt{\frac{8 \pi^2 m E_n}{h^2}} = \frac{n\pi}{L} \quad | \quad n \in \mathbb{N}, \quad \ ^2
$$
$$
\frac{8 \pi^2 m E_n}{h^2} = \left(\frac{n\pi}{L}\right)^2, \quad \cdot \frac{h^2}{8 \pi^2 m}
$$
$$
E_n = \left(\frac{n\pi}{L}\right)^2 \frac{h^2}{8 \pi^2 m} 
$$
\begin{equation} \label{energy_n}
\therefore E_n = \frac{h^2}{8mL^2} n^2
\end{equation}

\subsection{2}
Se relevanta grafbilder i \emph{imgs/}.

\subsection{3}

Vi vet sedan tidigare $\psi_n(x)$:
$$
\psi_n(x) = \sqrt{ \frac{2}{L} } \sin \left( \sqrt{\frac{8 \pi^2 m E_n}{h^2}} x \right)
$$

och i uppgiften får vi att:
$$
\Psi_n(x, t) = \psi_n(x) \cdot e^{-i \frac{E_n}{\hbar}t }
$$
$$
\implies \Psi_n(x, t) = \sqrt{ \frac{2}{L} } \sin \left( \sqrt{\frac{8 \pi^2 m E_n}{h^2}} x \right) \cdot e^{-i \frac{E_n}{\hbar}t } =
$$
$$
= \sqrt{ \frac{2}{L} } \sin \left( \sqrt{\frac{8 \pi^2 m E_n}{h^2}} x \right) \cdot e^{-i \frac{E_n}{ \frac{h}{2\pi} }t } 
= \sqrt{ \frac{2}{L} } \sin \left( \sqrt{\frac{8 \pi^2 m E_n}{h^2}} x \right) \cdot e^{-i \frac{2 E_n \pi}{h}t } 
$$
$$
\therefore \Psi_n(x,t) = \sqrt{ \frac{2}{L} } \left(e^{-i \frac{2 E_n \pi}{h}t }\right) \sin \left( \sqrt{\frac{8 \pi^2 m E_n}{h^2}} x \right) 
$$

Vi behöver nu bara normalisera integralen sådan att den alltid blir $1$ för följande:
$$
\Psi_{1,2}(x, t) = A \left( \psi_1(x)e^{-i \frac{2\pi E_1}{h}t } + \psi_2(x)e^{-i \frac{2\pi E_2}{h}t } \right)
$$
$$
\int_0^L |\Psi_{1,2}(x, t)|^2 dx\ = 1
$$

Eftersom vi integrerar med respekt till $x$ tyder det på att $t$ är en konstant och vi kan därmed skriva om tidsfaktorn som $z_i$:
$$
\int_0^L |\Psi_{1,2}(x, t)|^2 dx\ = \int_0^L \left(A |\psi_1(x)z_1 + \psi_2(x)z_2| \right)^2 dx\
$$
$$
= A^2 \int_0^L |\psi_1(x)z_1 + \psi_2(x)z_2|^2 dx\ = 1.0 \quad | \quad z_i \in \mathbb{C}
$$

Eftersom integranden är magnituden av ett komplext tal ges det att man kan se den som en vektor av $\mathbb{R}^2$ vilket i sin tur menas att magnituden är dess längd. Vi kan därmed skriva om magnituden med hjälp av \emph{pythagoras sats} (väljer också att skriva bort "$(x)$" från vågfunktionerna):

$$
|\psi_1(x)z_1 + \psi_2(x)z_2| = \sqrt{ \left( \psi_1 \emph{Re}(z_1) + \psi_2 \emph{Re}(z_2) \right)^2 + \left( \psi_1 \emph{Im}(z_1) + \psi_2 \emph{Im}(z_2) \right)^2 }
$$

Eftersom denna magnituden är i kvadrat får vi:
$$
|\psi_1(x)z_1 + \psi_2(x)z_2|^2 = \left( \psi_1 \emph{Re}(z_1) + \psi_2 \emph{Re}(z_2) \right)^2 + \left( \psi_1 \emph{Im}(z_1) + \psi_2 \emph{Im}(z_2) \right)^2 
$$
$$
= \left( \psi_1 \emph{Re}(z_1) + \psi_2 \emph{Re}(z_2) \right)^2 + \left( \psi_1 \emph{Im}(z_1) + \psi_2 \emph{Im}(z_2) \right)^2 
$$
$$
= ( \psi_1^2 \emph{Re}^2(z_1) + 2\psi_1 \psi_2 \emph{Re}(z_1)\emph{Re}(z_2) + \psi_2^2 \emph{Re}^2(z_2) )^2 
$$
$$
+ ( \psi_1^2 \emph{Im}^2(z_1) + 2\psi_1 \psi_2 \emph{Im}(z_1)\emph{Im}(z_2) + \psi_2^2 \emph{Im}^2(z_2) )^2 
$$
$$
\dots
$$

Det här är dock onödigt då vi kan också använda oss av \emph{Diracs notation}. Även kallat för \emph{bra-ket notationen}:
$$
\braket{\Psi_{1,2}|\Psi_{1,2}} = 1
$$
$$
\ket{\Psi_{1,2}}
$$

\end{document}
